\section{Experiencia con Unity-Robotics-Hub}

Unity-Robotics-Hub fue la primera librería que se evaluó para ser utilizada para integrar ROS2 en Unity. Esta librería mostraba signos de robustez y de amplia utilización, lo cual indicaba que podía ser una opción viable.



No obstante, se encontraron serios problemas, ya que la librería presentaba fallos desde las versiones tempranas del código. Existe la posibilidad de que estos problemas se deban a algún tipo de incompatibilidad con la SDK de NextMind.



Se intentaron múltiples soluciones, como la modificación completa del código y el análisis de los registros de Unity. También se probaron las implementaciones de NextMind SDK y UnityRoboticsHub de manera independiente, observando que funcionaban correctamente por separado, pero no conjuntamente.



Es una lastima que Unity-Robotics-Hub no haya funcionado de manera óptima, ya que, a pesar de funcionar a través de un intermediario, el código resultante era más legible y la carga para Unity no era tan elevada como con la implementación actual, que debe manejar ROS2 en su totalidad, generando nodos dentro de Unity.
