\section{Metodologías de investigación}
\label{researchmethodology}

Este proyecto adopta una metodología de investigación empírica, implementada a través de pruebas con usuarios. La elección de un diseño empírico se basa en la naturaleza individualizada del uso del casco BCI, que requiere una calibración única para cada individuo. Esta metodología permite analizar las variaciones individuales y determinar su efecto en la calibración y el rendimiento del BCI.



Las pruebas se realizaron con una aplicación que he desarrollado para el control de dispositivos electromecánicos, en este caso un sistema Pan-Tilt con una cámara M5 Stack Timercam.



La aplicación de Pan-Tilt está diseñada para controlar los dos grados de libertad del sistema y ofrecer la visualización en tiempo real del entorno desde la cámara que posee.



Las pruebas para el sistema Pan-Tilt se desarrollaron en un entorno que cuenta con un recorrido de códigos QR dispuestos en distintos planos.



\subsection{Participantes y reclutamiento}

Para este experimento, se reclutaron 16 participantes a través de la universidad y de amigos y conocidos. Se realizó un esfuerzo para asegurar la diversidad de la muestra, lo cual enriquece los resultados al permitir la observación de una gama más amplia de reacciones individuales al BCI. No se proporcionaron incentivos más allá de la oportunidad de experimentar con tecnología innovadora.



\subsection{Procedimiento experimental}

El experimento consta de dos partes principales: la calibración y la prueba. La calibración del BCI se realiza al principio y dura 55 segundos. Este proceso consiste en mostrar una serie de patrones o estímulos visuales a ciertas frecuencias para que el BCI pueda reconocer y calibrarse a las respuestas cerebrales individuales.

Una vez calibrado el BCI, se inicia la prueba, que tiene una duración máxima de diez minutos. Para el control del tiempo se ha incorporado un cronómetro en la aplicación. Los participantes deben recorrer un circuito de códigos QR dispuestos en distintos planos desde el inicio hasta donde puedan llegar o hasta que lo finalicen.

Los experimentos se realizarán en tres condiciones diferentes: sin luz en interior, con luz en interior y con luz en exterior. Este enfoque permite evaluar la funcionalidad de la aplicación en diversas condiciones de iluminación, ya que este factor parece influir en la precisión de la interfaz cerebro computador.



\subsection{Recopilación y análisis de datos}

Durante las pruebas, se lleva un seguimiento de comportamientos inesperados, especialmente en caso de fallos del BCI. También se controlará el tiempo de calibración del BCI, si los participantes finalizan la prueba y, en caso de ser así, el tiempo que les lleva hacerlo.



Los datos recopilados serán analizados a través de medidas descriptivas básicas, incluyendo el promedio. Además, se buscarán posibles correlaciones entre los diferentes factores, como la edad del participante y el tiempo de calibración del BCI.



\subsection{Consideraciones éticas}

En cuanto a la ética en la investigación, se ha garantizado el consentimiento informado, la privacidad y la confidencialidad de los participantes a través del cuestionario anteriormente mencionado que se entregó al final del experimento.



\subsection{Gestión de problemas}

Se espera que existan variaciones en los resultados entre los distintos individuos. No obstante, todas las pruebas se realizan de manera uniforme para mantener la coherencia del experimento. En caso de problemas o fallos en el BCI durante la experimentación, se ha incorporado un sistema de registro de fallos en la aplicación. Este sistema permite al participante registrar un fallo en la aplicación cuando detecte que el BCI ha fallado, proporcionando un registro contable de los problemas encontrados durante la prueba.