\section{Contexto y justificación}

Una interfaz cerebro computador (BCI, por sus siglas en inglés) establece una comunicación directa entre la actividad eléctrica del cerebro y dispositivos externos, como ordenadores o dispositivos electromecánicos. Estas interfaces se clasifican en diferentes niveles de invasividad según la proximidad de los electrodos y el tejido cerebral, pudiendo ser no invasivas, parcialmente invasivas o invasivas.

La integración de interfaces cerebro computador en el control de dispositivos presenta numerosas ventajas y aplicaciones. Uno de las ventajas más destacadas es su capacidad para ayudar a personas con problemas de movilidad a realizar tareas cotidianas, como el manejo de una silla de ruedas, el control una prótesis o la interacción con sistemas domóticos.

En el marco del presente Trabajo de Fin de Grado en Ingeniería Informática, se enfoca en el estudio e integración de una interfaz cerebro computador específica, denominada NextMind, para el control dispositivos electromecánicos. El objetivo principal de este trabajo es la exploración y la evaluación de las posibles aplicaciones beneficiosas que esta integración podría aportar.
