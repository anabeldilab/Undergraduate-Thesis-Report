\section{Contexto y justificación}

Una interfaz cerebro computador (BCI, por sus siglas en inglés) establece una comunicación directa entre la actividad eléctrica del cerebro y dispositivos externos, como ordenadores o dispositivos electromecánicos. Estas interfaces se clasifican en diferentes niveles de invasividad según la proximidad de los electrodos y el tejido cerebral, pudiendo ser no invasivas, parcialmente invasivas o invasivas.

\bigskip

La integración de interfaces cerebro computador en el control de dispositivos presenta numerosas ventajas y aplicaciones. Uno de las ventajas más destacadas es su capacidad para ayudar a personas con problemas de movilidad a realizar tareas cotidianas, como el manejo de una silla de ruedas, el control una prótesis o la interacción con sistemas domóticos.

\bigskip

En el marco del presente Trabajo de Fin de Grado en Ingeniería Informática, se enfoca en el estudio e integración de una interfaz cerebro computador específica, denominada NextMind, para el control dispositivos electromecánicos. 

\bigskip

NextMind, una startup de neurotecnología que ganó el premio a la Mejor Innovación en CES 2020, el 8 de diciembre de 2020, NextMind sacó su kit de desarrollo para su dispositivo BCI llamado por el mismo nombre. \cite{BusinessWire2020} Posteriormente, fue adquirida por Snap Inc, una reconocida empresa de tecnología y cámaras estadounidense. \cite{SnapInc} 

\bigskip

NextMind destaca por su robustez en la detección del estímulo en la actividad cerebral, lo cual lo convierte en una herramienta confiable para la captura precisa de las señales cerebrales. Además, su asequible precio lo hace accesible para investigadores y desarrolladores. Asimismo, su relativa comodidad de uso garantiza una experiencia satisfactoria para los usuarios durante sesiones prolongadas de interacción con el BCI.

\bigskip

Dada la creciente importancia de las interfaces cerebro-computadora en numerosas aplicaciones, el objetivo principal de este trabajo es investigar y desarrollar una aplicación en el que se use NextMind para el control de dispositivos electromecánicos, con el fin de explorar su viabilidad y potencial en esta área.