\section{Objetivos}

Los objetivos del presente trabajo de fin de grado son los siguientes:

\begin{enumerate}
\item Desarrollo de una interfaz de usuario para BCI: Esta aplicación permitirá la creación de una interfaz de usuario basada en BCI para el control de dispositivos electromecánicos, como por ejemplo un servomotor o un interruptor. La aplicación proporcionará las herramientas UI necesarias para una interacción fluida entre el usuario y los dispositivos, mediante la detección y el procesamiento de las señales cerebrales.

\item Adaptación a un caso de uso concreto: La aplicación se adaptará para el control de un dispositivo específico, como un sistema de apuntado (pan tilt) o un robot móvil. Este objetivo tomará en consideración las necesidades y requisitos del dispositivo seleccionado.

\item Evaluación de la aplicación: Se realizará una recopilación y análisis de las experiencias de los usuarios al interactuar con el BCI y los dispositivos electromecánicos controlados. Esta evaluación se realizará a través de entrevistas o cuestionarios que recogerán las opiniones de los usuarios y su percepción de la eficacia y facilidad de uso de la aplicación. Los datos recolectados serán analizados de manera cuantitativa y cualitativa para identificar áreas de mejora y de futuro desarrollo. Dentro de la aplicación habrán teclas a modo de flags/contadores para señalizar fallos y eventos ocurridos durante la prueba.
\end{enumerate}