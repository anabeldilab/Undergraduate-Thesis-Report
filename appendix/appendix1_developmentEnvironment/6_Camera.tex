\section{Cámara}

La cámara utilizada es una M5 Stack Timercam, como se menciona en la sección \ref{section:camara}.

\subsection{Instalación del entorno de trabajo}
\label{section:cameraenvironmentinstallation}

El software de la cámara es específico y solo es compatible con la versión ESP-IDF v4.0.1. Para garantizar una correcta instalación, es necesario clonar esta versión específica del repositorio.

\begin{verbatim}
git clone -b v4.0.1 --recursive 
https://github.com/espressif/esp-idf.git esp-idf-v4.0.1
\end{verbatim}

Posteriormente, es necesario instalar el entorno de desarrollo descargado para poder interactuar con él.

\begin{verbatim}
~/esp_idf_4.0/esp-idf-v4.0.1$ ./install.sh
\end{verbatim}

Para poder compilar el código de la cámara o monitorizarla desde la terminal, entre otras acciones, siempre será necesario ejecutar el siguiente comando en el directorio ``~/esp\_idf\_4.0/esp-idf-v4.0.1'':

\begin{verbatim}
source export.sh
\end{verbatim}

Es importante mencionar que no se puede ejecutar este comando después de haber hecho un source export.sh de otro ESP-IDF.

Para facilitar el uso de estos comandos, se han creado los siguientes alias que se deben añadir al archivo .bashrc:

\begin{verbatim}
alias get_idf='. $HOME/esp/esp-idf/export.sh'
alias get_idf4='. $HOME/esp_idf_4.0/esp-idf-v4.0.1/export.sh'
\end{verbatim}
