\newpage 
\thispagestyle{empty}

\begin{abstract}
{\em

En este trabajo se explora la integración de una interfaz cerebro computador (BCI) para el control de dispositivos electromecánicos. El objetivo principal es explorar y evaluar las aplicaciones beneficiosas de esta interfaz, especialmente en personas con movilidad reducida. Se logró implementar con éxito la interfaz BCI en un sistema pan-tilt y un robot, y se llevaron a cabo pruebas con participantes utilizando una interfaz gráfica específicamente desarrollada para evaluar la efectividad y la experiencia de usuario. Los resultados obtenidos indican que las interfaces cerebro computador son una herramienta útil para la inclusión de personas con limitaciones de movilidad, permitiéndoles lograr una mayor independencia en el control de dispositivos electromecánicos específicos.
}
\bigskip

\begin{palabrasClave}

Interfaz cerebro computador, control de dispositivos electromecánicos, movilidad reducida, inclusión, experiencia de usuario
\end{palabrasClave}

\end{abstract}